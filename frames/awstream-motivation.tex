\begin{frame}{Network Resource Constraints}
  \vspace{1em}
  \metroset{block=fill}
  \only<2,3>{
    \begin{columns}
      \column{0.5\textwidth}
      \begin{alertblock}{Demand}
        Huge Data Volume at the Edge
      \end{alertblock}
      \column{0.5\textwidth}
      \begin{block}{Resource}
        Insufficient WAN Bandwidth
      \end{block}
    \end{columns}
  }
  \only<4>{
    \begin{columns}
      \column{0.5\textwidth}
      \begin{block}{Demand}
        Huge Data Volume at the Edge
      \end{block}
      \column{0.5\textwidth}
      \begin{alertblock}{Resource}
        Insufficient WAN Bandwidth
      \end{alertblock}
    \end{columns}
  }
  \only<1,5->{
    \begin{columns}

      \column{0.5\textwidth}
      \begin{block}{Demand}
        Huge Data Volume at the Edge
      \end{block}
      \column{0.5\textwidth}
      \begin{block}{Resource}
        Insufficient WAN Bandwidth
      \end{block}
    \end{columns}
  }    

  \vspace{1em}

  \alt<2>{
    \begin{itemize}
    \item Video surveillance, 3 mbps per camera~\cite{amerasinghe2009h}
    \item Electrical grid monitoring, 1.4 million data points per second~\cite{andersen2016btrdb}
    \item Machine logs, 25 TB daily at Facebook (2009)
    \end{itemize}
  }

  \alt<3>{
    \textit{$\ldots$ Dropcam, a WiFi video-streaming camera and associated cloud backend
    service for storing and watching the resulting video. Dropcam has the fewest
    clients (2,940) $\ldots$. Yet, each client uses roughly 2.8 GB a week and
    uploads nearly 19 times more than they download, implying that Dropcam users
    do not often watch what they record.}

    Large-scale Measurements of Wireless Network Behavior \\
    \cite{biswas2015large}
  }

  \alt<4>{
    \begin{figure}
      \centering
      \includegraphics[width=\textwidth]{figures/aws-variation.pdf}
      \caption{Bandwidth variations throughout the day between Amazon EC2 sites.}
    \end{figure}
  }  

  \visible<5>{
    \begin{figure}
      \begin{subfigure}{0.49\textwidth}
        \centering
        \includegraphics[width=\textwidth]{figures/cisco.pdf}
        \caption{Cisco Project Data Growth~\cite{cisco2013zettabyte}}
      \end{subfigure}
      \hfill
      \begin{subfigure}{0.49\textwidth}
        \includegraphics[width=\textwidth]{figures/telegeography.pdf}
        \caption{Network backbone capacity growth slows
          down~\cite{global2016telegeography}}
      \end{subfigure}
    \end{figure}
  }  
\end{frame}

\begin{frame}{Fidelity vs.\,Freshness}
  When the network resource is not sufficient:

  \begin{itemize}
  \item TCP ensures data delivery, but hurts latency
  \item UDP sends as fast as possible, uncontrolled packet loss
  \item Developer heuristics are sub-optimal
    \only<1>{
      \begin{itemize}
      \item JetStream~\cite{rabkin2014aggregation} uses manual policy
      \item ``if bandwidth is insufficient, switch to sending images at 75\% fidelity,
        then 50\% if still not enough''
      \end{itemize}
    }
  \item Application-specific optimizations don't generalize
    \only<1>{
      \begin{itemize}
      \item Video streaming often aims at Quality of Experience (limited degradation dimension, e.g. maintain
        25FPS)
      \item More on next slide
      \end{itemize}
    }
  \end{itemize}

  \only<2>{
    \begin{figure}
      \centering
      \includegraphics[width=0.8\columnwidth]{figures/fidelity-freshness.pdf}
    \end{figure}
  }
\end{frame}

\captionsetup[figure]{labelformat=empty,font=scriptsize,labelfont=scriptsize}
\begin{frame}{Application-specific Optimizations Don't Generalize}
  \pgfplotstableread[row sep=\\,col sep=&]{
    Frame Rate & Bandwidth (normalized) & Accuracy \\
    30 & 100 & 100 \\
    10 & 40 & 92 \\
     5 & 21 & 90 \\
     3 & 13 & 87 \\
     2 & 9 & 84 \\
  }\stationaryframerate
  \pgfplotstableread[row sep=\\,col sep=&]{
    Resolution & Bandwidth (normalized) & Accuracy \\
    1080p & 100 & 100 \\
    900p & 79 & 87 \\
    720p & 54 & 84 \\
    540p & 29 & 71 \\
    360p & 17 & 11 \\
  }\stationaryresolution

  \begin{columns}[c]
    \column{0.4\textwidth}
    \vspace{2em}
    \begin{figure}
      \centering
      \includegraphics[width=\linewidth]{figures/mot-1.pdf}
      \caption{t=0s, small target in far-field views}
    \end{figure}
    \vspace{-1em}
    \begin{figure}
      \centering
      \includegraphics[width=\linewidth]{figures/mot-2.pdf}
      \caption{t=1s, small difference}
    \end{figure}

    \column{0.6\textwidth}
    \vspace{1.5em}

    \begin{tikzpicture}
      \tikzstyle{every node}=[font=\footnotesize]
      \begin{axis}[
        ybar,
        bar width               = .4cm,
        width                   = 1.1\textwidth,
        height                  = 0.4\textheight,
        legend style            = {at = {(0.5, 1.4)}, anchor = north,legend columns = -1},
        symbolic x coords       = {30, 10, 5, 3, 2},
        xtick                   = data,
        enlarge x limits        = 0.15,
        ymin                    = 0,
        ymax                    = 130,
        xlabel                  = {Frame Rate},
        nodes near coords,
        nodes near coords align = {vertical},
        ]
        \addplot table[x=Frame Rate,y=Bandwidth (normalized)]{\stationaryframerate};
        \addplot table[x=Frame Rate,y=Accuracy]{\stationaryframerate};
        \legend{Bandwidth (normalized), Accuracy}
      \end{axis}
    \end{tikzpicture}

    \vspace{1em}

    \begin{tikzpicture}
      \tikzstyle{every node}=[font=\footnotesize]
      \begin{axis}[
        ybar,
        bar width               = .4cm,
        width                   = 1.1\textwidth,
        height                  = 0.4\textheight,
        symbolic x coords       = {1080p, 900p, 720p, 540p, 360p},
        xtick                   = data,
        enlarge x limits        = 0.15,
        nodes near coords,
        nodes near coords align = {vertical},
        ymin                    = 0,
        ymax                    = 130,
        xlabel                  = {Resolution},
        ]
        \addplot table[x = Resolution, y = Bandwidth (normalized)]{\stationaryresolution};
        \addplot table[x = Resolution, y = Accuracy]{\stationaryresolution};
        \legend{}
      \end{axis}
    \end{tikzpicture}
  \end{columns}
\end{frame}


\begin{frame}{Application-specific Optimizations Don't Generalize}
  \captionsetup[figure]{labelformat=empty,font=scriptsize,labelfont=scriptsize}

  \pgfplotstableread[row sep=\\,col sep=&]{
    Frame Rate & Bandwidth (normalized) & Accuracy \\
    30 & 100 & 100 \\
    10 & 65 & 64 \\
    5 & 46 & 32 \\
    3 & 34 & 18 \\
    2 & 27 & 10 \\
  }\mobileframerate
  \pgfplotstableread[row sep=\\,col sep=&]{
    Resolution & Bandwidth (normalized) & Accuracy \\
    1080p & 100 & 100 \\
    900p & 69 & 99 \\
    720p & 49 & 97 \\
    540p & 33 & 93 \\
    360p & 22 & 87 \\
  }\mobileresolution

  \begin{columns}[c]
    \column{0.4\textwidth}
    \vspace{2em}
    \begin{figure}
      \centering
      \includegraphics[width=\linewidth]{figures/darknet-1.pdf}
      \caption{t=0s, nearby and large targets}
    \end{figure}
    \vspace{-1em}
    \begin{figure}
      \centering
      \includegraphics[width=\linewidth]{figures/darknet-2.pdf}
      \caption{t=1s, large difference}
    \end{figure}

    \column{0.6\textwidth}
    \vspace{1.5em}

    \begin{tikzpicture}
      \tikzstyle{every node}=[font=\footnotesize]
      \begin{axis}[
        ybar,
        bar width               = .4cm,
        width                   = 1.1\textwidth,
        height                  = 0.4\textheight,
        legend style            = {at = {(0.5, 1.4)}, anchor = north,legend columns = -1},
        symbolic x coords       = {30, 10, 5, 3, 2},
        xtick                   = data,
        enlarge x limits        = 0.15,
        ymin                    = 0,
        ymax                    = 130,
        xlabel                  = {Frame Rate},
        nodes near coords,
        nodes near coords align = {vertical},
        ]
        \addplot table[x=Frame Rate,y=Bandwidth (normalized)]{\mobileframerate};
        \addplot table[x=Frame Rate,y=Accuracy]{\mobileframerate};
        \legend{Bandwidth (normalized), Accuracy}
      \end{axis}
    \end{tikzpicture}

    \vspace{1em}

    \begin{tikzpicture}
      \tikzstyle{every node}=[font=\footnotesize]
      \begin{axis}[
        ybar,
        bar width               = .4cm,
        width                   = 1.1\textwidth,
        height                  = 0.4\textheight,
        symbolic x coords       = {1080p, 900p, 720p, 540p, 360p},
        xtick                   = data,
        enlarge x limits        = 0.15,
        nodes near coords,
        nodes near coords align = {vertical},
        ymin                    = 0,
        ymax                    = 130,
        xlabel                  = {Resolution},
        ]
        \addplot table[x = Resolution, y = Bandwidth (normalized)]{\mobileresolution};
        \addplot table[x = Resolution, y = Accuracy]{\mobileresolution};
        \legend{}
      \end{axis}
    \end{tikzpicture}
  \end{columns}
\end{frame}

\begin{frame}{Making Adaptation Practical is Challenging}
  \vspace{1em}

  \metroset{block=fill}

  \begin{block}{Goal}
    Minimize bandwidth while maximizing application accuracy
  \end{block}

  \textbf{Challenges}:

  \begin{enumerate}
    \visible<2->{\item Application-specific optimizations don't generalize.}
    \visible<5->{
      \begin{itemize}
      \item API. \texttt{maybe} operators to express adaptation.
      \end{itemize}
    }    
    \visible<3->{
    \item It requires expertise and manual work to explore multidimensional adaptation.
    }
    \visible<6->{
      \begin{itemize}
      \item Profiling: automatically learn Pareto-optimal strategy with
        multi-dimensional exploration.
      \end{itemize}
    }
    \visible<4->{
    \item The adaptation happens at the runtime.
    }
    \visible<7->{
      \begin{itemize}
      \item Engineering an adaptation system to balance latency and accuracy.
      \end{itemize}
    }
  \end{enumerate}
\end{frame}

\begin{frame}{System Architecture Overview}
  \centering
  \includegraphics[width=0.8\linewidth]{figures/arch.pdf}\\
  \begin{tikzpicture}
    \draw[dashed] (0,0) -- (10,0);
  \end{tikzpicture}
  \centering
  \includegraphics[width=0.7\linewidth]{figures/arch-2.pdf}
\end{frame}

%%% Local Variables:
%%% mode: latex
%%% TeX-master: "../talk"
%%% TeX-engine: xetex
%%% End:
